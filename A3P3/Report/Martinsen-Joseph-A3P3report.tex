%% LyX 2.2.2 created this file.  For more info, see http://www.lyx.org/.
%% Do not edit unless you really know what you are doing.
\documentclass[english]{article}
\usepackage{mathptmx}
\usepackage{helvet}
\usepackage{courier}
\usepackage[T1]{fontenc}
\usepackage[latin9]{inputenc}
\usepackage{geometry}
\geometry{verbose,tmargin=2.5cm,bmargin=2.5cm,lmargin=2.5cm,rmargin=2.5cm,headheight=0cm,headsep=0cm}

\makeatletter
\@ifundefined{date}{}{\date{}}
\makeatother

\usepackage{babel}
\begin{document}

\title{The Programming Assignment Report Instructions\\
CSCE 221}
\maketitle
\begin{enumerate}
\item The description of an assignment problem.\\ \ \\

The purpose of this assignment was to create a C++ program that would allow a user to enter an expression with or without variables in infix form. The simple calculator would then evaluate the expression by utilizing a stack, a que, and postfix form. The stack and que also used the templated linked list class created in part 1. \\ \ \\

\item The description of data structures and algorithms used to solve the
problem.

\begin{enumerate}
\item Provide definitions of data structures by using Abstract Data Types
(ADTs) \\ \ \\

\begin{itemize}
  \item Templated Doubly Linked List
  \item Templated Linked Que
  \item Templated Linked Stack
  \item String
  \item Vector
  \item Token
\end{itemize} \ \\
\item Write about the ADTs implementation in C++. \\ \ \\
Notice that during the evaluation process of an expression the stack
data structure should be used.

\begin{itemize}
  \item Templated Doubly Linked List
  \item Templated Linked Que
  \item Templated Linked Stack
  \item String
  \item Vector
  \item Token
\end{itemize} \ \\


\item Describe algorithms used to solve the problem.\\ \ \\
\begin{itemize}
  \item $Parser::toPostfix()$
  \item $Token::get_operator_weight()$
  \item $Evaluator::evaluate$
  \item $Evaluator::getValue()$
\end{itemize} \ \\

\item Analyze the algorithms according to assignment requirements. \\ \ \\
\begin{itemize}
  \item $Parser::toPostfix()$
  \item $Token::get_operator_weight()$
  \item $Evaluator::evaluate$
  \item $Evaluator::getValue()$
\end{itemize} \ \\

\end{enumerate}
\item A C++ organization and implementation of the problem solution 
\begin{enumerate}
\item Provide a list and description of classes or interfaces used by a
program such as classes used to implement the data structures or exceptions.\\ \ \\

\begin{itemize}
  \item Parser
  \item Evaluator
  \item LinkedQue
  \item LinkedStack
  \item RuntimeException
\end{itemize} \ \\

\item Include in the report the class declarations from a header file (.h)
and their implementation from a source file (.cpp). \vfill{}
\item Provide features of the C++ programming paradigms like Inheritance
or Polymorphism in case of object oriented programming, or Templates
in the case of generic programming used in your implementation. \\ \ \\

Templated
\begin{itemize}
  \item LinkedQue
  \item LinkedStack
\end{itemize}

Inheritance
\begin{itemize}
  \item RuntimeException
\end{itemize}\ \\
\end{enumerate}
\item A user guide description how to navigate your program with the instructions
how to: 
\begin{enumerate}
\item compile the program: specify the directory and file names, etc.\vfill{}
\item run the program: specify the name of an executable file. \vfill{}
\pagebreak{}
\end{enumerate}
\item Specifications and description of input and output formats and files 
\begin{enumerate}
\item The type of files: keyboard, text files, etc (if applicable). \\ \ \\
No input or output files \\ \ \\

\item A file input format: when a program requires a sequence of input items,
specify the number of items per line or a line termination. Provide
a sample of a required input format. \vfill{}
\item Discuss possible cases when your program could crash because of incorrect
input (a wrong file name, strings instead of a number, or such cases
when the program expects 10 items to read and it finds only 9.)\vfill{}
\end{enumerate}
\item Provide types of exceptions and their purpose in your program.
\begin{enumerate}
\item logical exceptions (such as deletion of an item from an empty container,
etc.).\\ \ \\

\begin{itemize}
  \item QueueEmptyException
  \item StackEmptyException
\end{itemize} \ \\

\item runtime exception (such as division by $0$, etc.)\\ \ \\

\begin{itemize}
  \item $DivisionByZeroException()$ in evaluator.h
\end{itemize}\ \\

\end{enumerate}
\item Test your program for correctness using valid, invalid, and random
inputs (e.g., insertion of an item at the beginning, at the end, or
at a random place into a sorted vector). Include evidence of your
testing, such as an output file or screen shots with an input and
the corresponding output. \vfill{}
\end{enumerate}

\end{document}
