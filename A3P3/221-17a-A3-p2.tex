%% LyX 2.2.2 created this file.  For more info, see http://www.lyx.org/.
%% Do not edit unless you really know what you are doing.
\documentclass[english]{article}
\usepackage{mathptmx}
\usepackage[T1]{fontenc}
\usepackage[latin9]{inputenc}
\usepackage{geometry}
\geometry{verbose,tmargin=1in,bmargin=1in,lmargin=1in,rmargin=1in,headheight=0in,headsep=0in}
\pagestyle{empty}

\makeatletter

%%%%%%%%%%%%%%%%%%%%%%%%%%%%%% LyX specific LaTeX commands.
%% For printing a cirumflex inside a formula
\newcommand{\mathcircumflex}[0]{\mbox{\^{}}}


%%%%%%%%%%%%%%%%%%%%%%%%%%%%%% User specified LaTeX commands.
\newcommand{\AAS}{\textbf{Assignment}}
\newcommand{\vdist}{\vspace*{10em}}
\newcommand{\dist}{\hspace{2em}}
\newcommand{\ddist}{\hspace{4em}}
\newcommand{\dddist}{\hspace{6em}}

\makeatother

\usepackage{babel}
\begin{document}
\begin{center}
\textbf{\large{}Assignment~3 \textendash{} Part 2 \& 3}
\par\end{center}{\large \par}

\begin{center}
\emph{Part 2 and 3 due March 19 at midnight}
\par\end{center}

\begin{center}
\emph{\smallskip{}
}
\par\end{center}

We use the same instructions for grading as for previous assignments.

\medskip{}
\textbf{Problem.} 
\begin{itemize}
\item Write a C++ program for a simple calculator based on expression postfix
form.
\begin{enumerate}
\item Part 2: implement templated stack and queue data structures using
the templated linked list class. 
\item Part 3:
\begin{enumerate}
\item implement the class \texttt{parser }with a function \texttt{to\_postfix()}
to translate input infix form of an algebraic expression into its
postfix form using stack and queue as the auxiliary data structures,
see lecture notes for the algorithm.
\item implement the class \texttt{evaluator} with a function \texttt{evaluate()}
to get the value of an algebraic expression based on its postfix form.
Notice that during the evaluation process of an expression the stack
data structure should be used.
\end{enumerate}
\end{enumerate}

\item Useful hints.
\begin{itemize}
\item Use the templated linked list class from Part 1 of the assignment
to implement the templated versions of the queue and stack data structures.
Test the correctness of the queue using string type and floating point
type (type \texttt{double}). 
\item The infix expression will consist of a valid combination of the following
operators: $(,),+,-,*,/,\mathcircumflex$. The operator $\mathcircumflex$
stands for the exponential operator (raising to the power).
\item Operands include 26 variables, namely, $a$ through $z$. Values for
the operands can be entered from the keyboard (interactively).  
\item Throw the exception ``\texttt{Invalid input}'' in the case when
an invalid name of a variable or incorrect operator is used.
\item Read the expression from the input token by token until the termination
character \texttt{\#} of the expression is found. Then your program
should be able to display the infix queue. Write a string tokenizer
function to extract tokens from the input string.
\item Print out the postfix queue. 
\item Evaluate the obtained postfix expression using the algorithm described
in class. Replace variables $a,..,z$ by associated values. 
\end{itemize}
\item The basic user menu for this program
\begin{enumerate}
\item Read an infix expression from the keyboard 
\item Check whether or not parenthesis are balanced correctly
\item Display a correct infix expression on the screen or a message that
the expression is invalid
\item Convert infix form to its postfix form and display a postfix queue
on the screen 
\item Evaluate postfix form of the expression for floating point values
entered from the keyboard
\item Display the value of an algebraic expression on the screen 
\end{enumerate}
\item What to include in the assignment report?
\begin{enumerate}
\item Provide the design of your program for the parts 2 and 3 of the assignment.
Write about the relation between classes and justification why you
chose them. 
\item Describe each class private and public members, your algorithms and
their implementations.
\item Write the names of the templated classes of your assignment. Which
types have you used to test them for correctness?
\item Describe all tests done to verify correctness of your program for
the parts 2 and 3 of the assignment. Give an explanation why you chose
such tests. Include your tests in your assignment report.
\end{enumerate}
\item Please submit to CSNet a tar file that includes:
\begin{itemize}
\item The templated version of the linked list, part 1 of the assignment.
\item The templated version of the queue and stack classes based on doubly
linked list (part 2)
\item The parser class (part 3)
\item The evaluator class (part 3)
\item The testing class with the main function and the menu of this program.
\end{itemize}
\end{itemize}

\end{document}
