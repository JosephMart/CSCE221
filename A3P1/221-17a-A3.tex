%% LyX 2.2.2 created this file.  For more info, see http://www.lyx.org/.
%% Do not edit unless you really know what you are doing.
\documentclass[english]{article}
\usepackage{mathptmx}
\usepackage{helvet}
\usepackage{courier}
\usepackage[T1]{fontenc}
\usepackage[latin9]{inputenc}
\usepackage{geometry}
\geometry{verbose,tmargin=1in,bmargin=1in,lmargin=1in,rmargin=1in,headheight=0in,headsep=0in}
\pagestyle{empty}
\usepackage{babel}
\usepackage[unicode=true]
 {hyperref}

\makeatletter

%%%%%%%%%%%%%%%%%%%%%%%%%%%%%% LyX specific LaTeX commands.
%% Because html converters don't know tabularnewline
\providecommand{\tabularnewline}{\\}

%%%%%%%%%%%%%%%%%%%%%%%%%%%%%% Textclass specific LaTeX commands.
\newenvironment{lyxcode}
{\par\begin{list}{}{
\setlength{\rightmargin}{\leftmargin}
\setlength{\listparindent}{0pt}% needed for AMS classes
\raggedright
\setlength{\itemsep}{0pt}
\setlength{\parsep}{0pt}
\normalfont\ttfamily}%
 \item[]}
{\end{list}}

%%%%%%%%%%%%%%%%%%%%%%%%%%%%%% User specified LaTeX commands.
\date{}

\makeatother

\begin{document}
\begin{center}
\textbf{\large{}CSCE 221 Assignment 3 Cover Page}\\
\bigskip{}
\par\end{center}

First Name~~~~~~~~~~~~~~~~~~~~~~~~~~~~~~~~~~Last
Name ~~~~~~~~~~~~~~~~~~~~~~~~UIN~~~~~~~~~~~~~~\bigskip{}

User Name ~~~~~~~~~~~~~~~~~~~~~~~~~~~~~E-mail
address~~~~~~~~~~~~~~~~~~~~~~~~~~~~~~\medskip{}

Please list all sources in the table below including web pages which
you used to solve or implement the current homework. If you fail to
cite sources you can get a lower number of points or even zero, read
more on Aggie Honor System Office website: \texttt{\href{http://aggiehonor.tamu.edu/}{http://aggiehonor.tamu.edu/}}\medskip{}
\medskip{}

\noindent \begin{flushleft}
\begin{tabular}{|c|c|c|c|c|}
\hline 
Type of sources  & ~~~~~~~~~~~~~~~~~~~~~~~ & ~~~~~~~~~~~~~~~~~~~~~~~~ & ~~~~~~~~~~~~~~~~~~~~~~~ & ~~~~~~~~~~~~~~~~~~~~~~~\tabularnewline
 &  &  &  & \tabularnewline
\hline 
People &  &  &  & \tabularnewline
 &  &  &  & \tabularnewline
\hline 
Web pages (provide URL)  &  &  &  & \tabularnewline
 &  &  &  & \tabularnewline
\hline 
Printed material &  &  &  & \tabularnewline
 &  &  &  & \tabularnewline
\hline 
Other Sources  &  &  &  & \tabularnewline
 &  &  &  & \tabularnewline
\hline 
\end{tabular}
\par\end{flushleft}

\medskip{}
\medskip{}

\noindent I certify that I have listed all the sources that I used
to develop the solutions/codes to the submitted work.

\noindent \emph{On my honor as an Aggie, I have neither given nor
received any unauthorized help on this academic work}.

\bigskip{}
\bigskip{}

\begin{tabular}{cccccc}
Your Name  & ~~~~~~~~~~~~~~~~~~~~~~~~~~~ &  & ~~~~~~~~~~~~~~~~~~~~~ & Date  & ~~~~~~~~~~~~~~~~~~~~\tabularnewline
\end{tabular}

\title{\newpage{}}

\title{\textbf{CSCE 221 Assignment 3 \textendash{} Part 1}}
\maketitle
\begin{center}
\emph{Part 1 due to CSNet by March 1 with demonstration in labs on
February 27/28. }
\par\end{center}

\section*{Objective}

This is an individual assignment which has three parts. 
\begin{enumerate}
\item Part 1: C++ implementation of a Doubly Linked List for \texttt{int}
type and next writing its templated version. The supplementary code
is provided (download it from the class website). 
\item Part 2: C++ implementation of queue and stack classes based on a templated
Doubly Linked List (implemented in Part 1. 
\item Part 3: C++ implementation of a simple calculator for evaluating an
algebraic expression based on its postfix form. The queue and stack
classes (implemented in Part 2) should be applied for obtaining the
postfix form and expression evaluation.
\end{enumerate}

\section*{Part 1: Implementation of Doubly Linked List}
\begin{itemize}
\item Download the program \texttt{221-A3-code.tar} from the course website.
Use the 7-zip software to extract the files in Windows, or use the
following command in Linux.
\begin{lyxcode}
tar~xfv~221-A3-code.tar
\end{lyxcode}
\item Two programs in separate folders are included.
\end{itemize}
\begin{enumerate}
\item ``Doubly Linked List'' for integers
\begin{enumerate}
\item Most code is extracted from the lecture slides. An exception structure
is defined to complete the program.
\item You need to complete the following functions in the \texttt{DoublyLinkedList.cpp}.
\begin{enumerate}
\item copy constructor
\item assignment operator
\item output operator
\item \texttt{insertFirst}
\item \texttt{removeFirst}
\item \texttt{first}
\end{enumerate}
Make sure the i. and ii. functions do a deep copy of the input list,
that is, copying each node one by one.
\item Type the following commands to compile the program
\begin{lyxcode}
make
\end{lyxcode}
\item The main program includes examples of creating doubly linked lists,
and demonstrates how to use them. Type the following command to execute.
\begin{lyxcode}
./run-dll
\end{lyxcode}
\end{enumerate}
\item Templated ``Doubly Linked List'' (general type)
\begin{enumerate}
\item Convert the doubly linked list in the previous program to a template,
so it creates lists of general types other than integer.
\item Read C++ slides, page 16-22 at \href{http://www.stroustrup.com/Programming/19_vector.ppt}{http://www.stroustrup.com/Programming/19\_{}vector.ppt}
\item Follow the instructions below:
\begin{enumerate}
\item Templates should be declared and defined in a \texttt{.h} file. Move
the content of \texttt{DoublyLinkedList.cpp} and \texttt{DoublyLinkedList.h}
to \texttt{TemplateDoublyLinkedList.h} 
\item Replace \texttt{int obj} by \texttt{T obj} in the class \texttt{DListNode}
so list nodes store general \texttt{T} objects instead of integers.
Later when a \texttt{DListNode} object is created, say, in the main
function, \texttt{T} can be specified as a \texttt{char}, a \texttt{string}
or a user-defined class.
\item To use a general type \texttt{T}, you must change each type declaration.
\begin{enumerate}
\item Replace variable declaration, input type and output type of functions
\texttt{int} by general type \texttt{T}, except for the \texttt{count}
variable.
\item Replace variable declaration, input type and output type of functions
\texttt{DListNode} by \texttt{DListNode<T>} , 
\item Replace variable declaration, input type and output type of functions
\texttt{DoublyLinkedList} by \texttt{DoublyLinkedList<T>}, including
the friend class declaration
\end{enumerate}
\item Assign general default value \texttt{T()} instead of the original
0 
\item To use general type \texttt{T} anywhere throughout the class \texttt{DListNode}
and \texttt{DoublyLinkedList}, you must declare (add) \texttt{template
<typename T>} before classes and the member functions defined outside
the class declaration
\item In each member function signature, replace \texttt{DoublyLinkedList::}
by \texttt{DoublyLinkedList<T>::}
\end{enumerate}
\item Type the following commands to compile the program.
\begin{lyxcode}
make
\end{lyxcode}
\item The main program includes examples of creating doubly linked lists
of ``strings'', and demonstrates how to use them. Type the following
command to execute.
\begin{lyxcode}
./run-tdll
\end{lyxcode}
\end{enumerate}
\end{enumerate}
\begin{itemize}
\item \textbf{Complexity Analysis}

Comment each class member function you implemented with its time complexity
using big-O notation. Specifically, comment on the loops. 
\item \textbf{What to submit to CSNet?}
\begin{itemize}
\item Your source code for \texttt{DoublyLinkedList} and \texttt{TemplateDoublyLinkedList}
\item Written report with complexity using big-O notation for all the functions.
\item The screenshots of testing all the cases should be included in your
reports.
\end{itemize}
\end{itemize}

\end{document}
