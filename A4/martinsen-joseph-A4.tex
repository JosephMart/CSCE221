%% LyX 2.2.2 created this file.  For more info, see http://www.lyx.org/.
%% Do not edit unless you really know what you are doing.
\documentclass[english]{article}
\renewcommand{\rmdefault}{cmr}
\renewcommand{\sfdefault}{cmss}
\usepackage{courier}
\usepackage[T1]{fontenc}
\usepackage[latin9]{inputenc}
\usepackage{geometry}
\usepackage{listings}
\geometry{verbose,tmargin=1in,bmargin=1in,lmargin=1in,rmargin=1in,headheight=0in,headsep=0in,footskip=0.5in}
\usepackage{color}
\usepackage{babel}
\usepackage[unicode=true]
 {hyperref}

\makeatletter

%%%%%%%%%%%%%%%%%%%%%%%%%%%%%% LyX specific LaTeX commands.
%% Because html converters don't know tabularnewline
\providecommand{\tabularnewline}{\\}

\makeatother

\usepackage{listings}
\renewcommand{\lstlistingname}{Listing}

\begin{document}
\begin{center}
\textbf{\textcolor{black}{\large{}CSCE 221 Assignment 4 Cover Page}}\textcolor{black}{}\\
\textcolor{black}{\bigskip{}
}
\par\end{center}

\textcolor{black}{First Name:\quad Joseph~~Last
Name ~Martinsen~UIN~323009961~~\bigskip{}
}

\textcolor{black}{User Name ~~~~josephmart~~~~E-mail
address~~josephmart@tamu.edu~~~\medskip{}
}

\textcolor{black}{Please list all sources in the table below including
web pages which you used to solve or implement the current homework.
If you fail to cite sources you can get a lower number of points or
even zero, read more on Aggie Honor System Office website: }\texttt{\textcolor{black}{\href{http://aggiehonor.tamu.edu/}{http://aggiehonor.tamu.edu/}}}\textcolor{black}{\medskip{}
\medskip{}
}
\noindent \begin{flushleft}
\textcolor{black}{}%
\begin{tabular}{|c|c|c|c|c|}
\hline 
\textcolor{black}{Type of sources } & \textcolor{black}{~~~~Stack Overflow~~} & \textcolor{black}{~~~~~~~~~~~~~~~~~~~~~~~~} & \textcolor{black}{~~~~~~~~~~~~~~~~~~~~~~~} & \textcolor{black}{~~~~~~~~~~~~~~~~~~~~~~~}\tabularnewline
 &  &  &  & \tabularnewline
\hline 
\textcolor{black}{People} &  &  &  & \tabularnewline
 &  &  &  & \tabularnewline
\hline 
\textcolor{black}{Web pages (provide URL) } &  &  &  & \tabularnewline
 &  &  &  & \tabularnewline
\hline 
\textcolor{black}{Printed material} &  &  &  & \tabularnewline
 &  &  &  & \tabularnewline
\hline 
\textcolor{black}{Other Sources } &  &  &  & \tabularnewline
 &  &  &  & \tabularnewline
\hline 
\end{tabular}
\par\end{flushleft}

\textcolor{black}{\medskip{}
\medskip{}
}

\noindent \textcolor{black}{I certify that I have listed all the sources
that I used to develop the solutions/codes to the submitted work.}

\noindent \textcolor{black}{\emph{On my honor as an Aggie, I have
neither given nor received any unauthorized help on this academic
work}}\textcolor{black}{.}

\textcolor{black}{\bigskip{}
\bigskip{}
}

\textcolor{black}{}%
\begin{tabular}{cccccc}
\textcolor{black}{Your Name } & \textcolor{black}{~~~~~Joseph~~} &  & \textcolor{black}{~~~Martinse~~~~~} & \textcolor{black}{Date } & \textcolor{black}{~~~03/30/2017~~~~}\tabularnewline
\end{tabular}\textcolor{black}{\pagebreak{}}

\begin{center}
\textcolor{black}{\large{}CSCE 221 \textemdash{} Programming Assignment
4, Spring 2017 (100 points)}
\par\end{center}{\large \par}

\begin{center}
\textbf{\textcolor{black}{Due: March 30th at 11:59 pm}}
\par\end{center}

\begin{center}\Huge\bfseries
Report\\
\vspace{15pt}
\small Joseph Martinsen \\
PA4\\
File Parsing and Regex
\end{center}

\noindent \begin{flushleft}
\textbf{\textcolor{black}{\large{}File Parsing and Regex \textendash{}
Assignment Description (100 points)}}
\par\end{flushleft}{\large \par}

A very common task in Computer Science is the reading in and parsing
of text. One of the most powerful tools at a programmer\textquoteright s
disposal to aid in this task is regex (which stands for REGular EXpressions). 

\begin{enumerate}
\item What is stored in \textquotedblleft \texttt{matches}\textquotedblright ? \\
  99
\item What does \textquotedblleft\texttt{\textbackslash{}d}\textquotedblright{} mean? \\
Digit

\item Modify the regex pattern to retrieve a two-digit number and the word
\texttt{thanks} in the string. Test your pattern for correctness. \\ Check the code

\item (25 points) Compile and run the following code, then answer the questions: 
\begin{enumerate}
\item What does \textquotedblleft\texttt{\textbackslash{}s\textbackslash{}S}\textquotedblright{}
mean? \\
\textquotedblleft\texttt{\textbackslash{}s}\textquotedblright{} means white space \\
\textquotedblleft\texttt{\textbackslash{}S}\textquotedblright{} means no white space.



\item What is stored in \texttt{matches{[}0{]}}? \\
\texttt{matches{[}0{]}}  is "<title>This is a title</title>"

\item Why is \texttt{matches{[}1{]}} different? \\
Because there are two regexs patterns in "R"(<title>([\textbackslash s\textbackslash S]+)</title>)"


\item Modify the regex pattern to retrieve only the items inside of the
header tag but not inside of the title tag. Test your pattern for
correctness.\\
Check the code
\end{enumerate}
\newpage{}
\item (40 points) Download the following text file: \href{http://courses.cs.tamu.edu/teresa/csce221/pdf/stroustrup.txt}{stroustrup.txt} 

Write a program using regex that will go through the text file and
print out the file name of every hyperlinked powerpoint file. (Hint1:
an HTML hyperlink uses the format \texttt{<a href=\textquotedblright ...>...</a>}.
Hint2. The powerpoint file extension is \texttt{.ppt}) \\
Check the code
\end{enumerate}

\begin{itemize}
\item Your C++ source code with the header block including: your name, user
name, section number and e-mail address
\lstinputlisting[language=C++,basicstyle={\small\ttfamily},showstringspaces=false]{./code/main.cpp}


\begin{itemize}
\item Description of input and output data. List all restrictions and assumptions
that you have imposed on your input data and program. \\

Input data is an html page and output is powerpoint names.

\item Write your regex patterns used for parsing the strings in the programs
above. Explain their syntax. \\

Part 3: Search for digit then digit for the first part. Next it looked for string literal \texttt{thanks} \\

Part 4: Had two regex patterns. First it searched for any white-space or non-white-space from title to head. Next regex pattern searched for any white-space or non-white-space from head to title. \\

Part 5: This part finds characters between <a href=" and " that end in ppt.
This search loops, once a match has been found, it will then continue searching.


\item What is the purpose of the functions \texttt{regex\_search()} and
\texttt{regex\_match()}. \\

\texttt{regex\_match()}\\
Checks whether there is a match between the regular expression e, and all of the character sequence \\

\texttt{regex\_search()}\\
Checks if there is a sub-sequence in the string that matches the regular expression.

\item Which C++ features or standard library classes have you used in your
program?
\begin{lstlisting}[language=C++,basicstyle={\small\ttfamily},showstringspaces=false]
#include <iostream>
#include <string>
#include <regex>
#include <fstream>
\end{lstlisting} 
\end{itemize}
\end{itemize}

\end{document}
